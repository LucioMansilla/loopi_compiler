
\documentclass{article}
\usepackage[utf8]{inputenc}
\usepackage{graphicx}
\usepackage{listings}

\title{Loopify - Taller de diseño de Software}
\author{Lucio Mansilla - Brenda Dichiara}
\date{\today}

\begin{document}
\maketitle

\tableofcontents

\section{Introducción}
El lenguaje incluye operaciones matemáticas básicas, estructuras de control de flujo, y declaraciones de funciones, entre otros. A continuación, se detallan las características, limitaciones y desafios que nos enfrentamos al desarrollar el lenguaje y su compilador.

\section{Descripción del Lenguaje}

Es un lenguaje tipado que admite la declaración de variables y métodos, asignaciones, llamadas a métodos, y estructuras de control como if, if-else y while.:

\subsection{Variables y Tipos de Datos}

El lenguaje soporta variables de tipo entero (\texttt{int}) y booleanas (\texttt{bool}). Las declaraciones de variables deben inicializarse con un valor.

\begin{verbatim}
int variable = 5;
bool boolean_var = true;
\end{verbatim}

\end{document}

