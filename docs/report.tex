\documentclass[12pt]{article}
\usepackage{listings}

\title{Manual de Usuario de Loopify}
\author{Tu nombre}
\date{Fecha de publicación}

\begin{document}
\maketitle

\tableofcontents

\section{Introducción}
Loopify es un lenguaje de programación diseñado para [...].

\section{Requisitos}
Para compilar y ejecutar programas escritos en Loopify, necesitarás [...]

\section{Tipos de datos}
Loopify soporta los siguientes tipos de datos:
\begin{itemize}
    \item \texttt{TINT}: Para representar números enteros.
    \item \texttt{TBOOL}: Para representar valores booleanos.
\end{itemize}

\section{Declaración de Variables}
Para declarar una variable en Loopify, se utiliza la siguiente sintaxis:
\begin{lstlisting}
TINT variable = expresion;
TBOOL variable = expresion;
\end{lstlisting}

\section{Operaciones}
Loopify soporta las siguientes operaciones:
\begin{itemize}
    \item Adición (\texttt{+})
    \item Multiplicación (\texttt{*})
\end{itemize}

\section{Sentencias}
\subsection{Asignación}
\begin{lstlisting}
variable = expresion;
\end{lstlisting}

\subsection{Retorno}
\begin{lstlisting}
RETURN expresion;
\end{lstlisting}

\section{Errores Comunes}
Aquí podemos describir algunos errores comunes que los usuarios podrían enfrentar y cómo resolverlos.

\section{Ejemplos}
\subsection{Programa Simple}
\begin{lstlisting}
TINT a = 1 + 2;
RETURN a;
\end{lstlisting}

\end{document}
